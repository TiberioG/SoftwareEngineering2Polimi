%%%%Basic service use case definition
	\textbf{ID}: \ucas{1} \\
	\textbf{Name}: Sign-Up \\
	\textbf{Actor}: Guest \\
	\textbf{Entry conditions}:
	\begin{enumerate}
		\item{A citizen who wants to use the service}
	\end{enumerate}
	\textbf{Event flow}:
	\begin{enumerate}
		\item{The guest reaches the registration page containing the relative form}
		\item{The guest fills up the form and clicks on "Sign up" to complete the process}
		\item{The system redirects the user to his profile page and sends a confirmation email}
	\end{enumerate}
	\textbf{Exit conditions}:
	\begin{itemize}
		\item{The guest has successfully registered in the system}
	\end{itemize}
	\textbf{Exceptions}:
	\begin{enumerate}
    		\item{The guest left an empty field or typed something wrong an error message is displayed and theuser is asked to fill the form again.}
 	   \end{enumerate}
	\rule{\linewidth}{0.4pt}
  %%%%%%%%%%%%%%%%%%%%%%%%%%%%%%%%%%%%%%%%%%%%%%%%%%%%%%%%%%%%%%%%%%%%%%%
	\textbf{ID}: \ucas{2} \\
	\textbf{Name}: Login \\
	\textbf{Actor}: User \\
	\textbf{Entry conditions}:
	\begin{enumerate}
		\item{The user has already registered}
	\end{enumerate}
	\textbf{Event flow}:
	\begin{enumerate}
		\item{The user reaches the login page containing the relative form}
		\item{The user types the username and password in the login form and click on ”Login” button}
		\item{The system redirects the user to the application homepage}
	\end{enumerate}
	\textbf{Exit conditions}:
	\begin{itemize}
		\item{The user has access to the application functionalities}
	\end{itemize}
	\textbf{Exceptions}:
	\begin{enumerate}
    		\item{Username and password didn’t correspond or the username didn’t exist, an error message isdisplayed and the user is asked to fill the login form again}
 	   \end{enumerate}
	\rule{\linewidth}{0.4pt}
  %%%%%%%%%%%%%%%%%%%%%%%%%%%%%%%%%%%%%%%%%%%%%%%%%%%%%%%%%%%%%%%%%%%%%%%
	\textbf{ID}: \ucas{3}  \\
	\textbf{Name}: Recover Password \\
	\textbf{Actor}: User \\
	\textbf{Entry conditions}:
	\begin{enumerate}
		\item{The user has already registered}
	\end{enumerate}
	\textbf{Event flow}:
	\begin{enumerate}
		\item{The user reaches the login page containing the relative form}
		\item{The user clicks on ”Password recovery” button and is redirected to the password recovery page.}
		\item{The user inserts his email and clicks on ”reset password”}
		\item{The system sends an email to the user with a link and instruction to reset the password}
		\item{The user chooses and types a new password and confirms}
		\item{The application check whether the entered password is strong enough or not}
		\item{The system redirects the user to the login page}
	\end{enumerate}
	\textbf{Exit conditions}:
	\begin{itemize}
		\item{The user has changed his password}
	\end{itemize}
	\textbf{Exceptions}:
	\begin{enumerate}
    		\item{The inserted email doesn't match any user in the database, it is displayed an error messageand the user is asked to retype a valid email.}
 	   \end{enumerate}
	\rule{\linewidth}{0.4pt}
  %%%%%%%%%%%%%%%%%%%%%%%%%%%%%%%%%%%%%%%%%%%%%%%%%%%%%%%%%%%%%%%%%%%%%%%
  \textbf{ID}: \ucas{4a} \\
  \textbf{Name}: Report a violation - taking picture \\
  \textbf{Actor}: User   \\
  \textbf{Entry conditions}:
  \begin{enumerate}
    \item{User is logged in}
  \end{enumerate}
  \textbf{Event flow}:
  \begin{enumerate}
    \item{User enters the section "Report a violation"}
    \item{System opens the camera of smartphone and ask user to take a picture of the violation}
    \item{The system reminds the user that violation and the licence plate of the veichle which is inviolation must be visible }
    \item{The user takes the picture }
    \item{The system shows the picture just taken }
    \item{The system asks the user: if there are other plates visible in the picture, which are not the oneof the veichle to be reported, use the finger to delete them }
    \item{The system enters in "brush tool mode" and the user covers the other licence plates}
    \item{When done, user press continue button}
    \item{The system sends the picture to the ALPR service which returns the string containing the platedecoded}
    \item{The system shows now on the screen the "report vioaltion form"}
  \end{enumerate}
  \textbf{Exit conditions}:
  \begin{enumerate}
    \item{User must continue to next \ucas{4b}}
  \end{enumerate}
  \textbf{Exceptions}:
  \begin{enumerate}
    \item{If no plate is found, the user has to repeat this use case, starting from taking the picture again}
    \item{If the ALPR service returns more than one plate, the user is informed that must delete the notrequired plates and the system goes agin to the "brush tool mode"}
    \item{If user doesn't continue to the next use case: e.g. presses exit button, or closes the app formore than 10 minutes, the picture taken is discarded}
  \end{enumerate}
  \rule{\linewidth}{0.4pt}
  %%%%%%%%%%%%%%%%%%%%%%%%%%%%%%%%%%%%%%%%%%%%%%%%%%%%%%%%%%%%%%%%%%%%%%%
  \textbf{ID}: \ucas{4b} \\
  \textbf{Name}: Report a violation - fill the form \\
  \textbf{Actor}: User   \\
  \textbf{Entry conditions}:
  \begin{enumerate}
    \item{User has successfully completed the precedent \ucas{4a}}
    \item{User is in the fill-form section of the app}
  \end{enumerate}
  \textbf{Event flow}:
  \begin{enumerate}
    \item{The system sends GPS location to the external service to get the complete address of the user}
    \item{the form is pre-filled with the address that is given by the external service  }
    \item{The user must choose from a list of violations the one referred to the picture taken which wantsto report. In the UI every row contains the name of the violation and a "info" button}
    \item{the user can choose to send the form or exit}
  \end{enumerate}
  \textbf{Exit conditions}:
  \begin{enumerate}
    \item{The violation is correctly inserted and stored}
  \end{enumerate}
  \textbf{Exceptions}:
  \begin{enumerate}
  \end{enumerate}
  \rule{\linewidth}{0.4pt}
  %%%%%%%%%%%%%%%%%%%%%%%%%%%%%%%%%%%%%%%%%%%%%%%%%%%%%%%%%%%%%%%%%%%%%%%
  \textbf{ID}: \ucas{4b1} \\
  \textbf{Name}: Report a violation - fill the form - violation infopage \\
  \textbf{Actor}: User   \\
  \textbf{Entry conditions}:
  \begin{enumerate}
    \item{User is in Use case \ucas{4b} }
    \item{User has pressed the "info" button of a violation from the list }
  \end{enumerate}
  \textbf{Event flow}:
  \begin{enumerate}
    \item{System shows a brief decriprion of the selected violation}
  \end{enumerate}
  \textbf{Exit conditions}:
  \begin{enumerate}
    \item{User goes back to Use case \ucas{4b}}
  \end{enumerate}
  \textbf{Exceptions}:
  \begin{enumerate}
  \end{enumerate}
  \rule{\linewidth}{0.4pt}
  %%%%%%%%%%%%%%%%%%%%%%%%%%%%%%%%%%%%%%%%%%%%%%%%%%%%%%%%%%%%%%%%%%%%%%%



	\textbf{ID}: \ucas{5a} \\
	\textbf{Name}: Mine information - street heatmap \\
	\textbf{Actor}: User  \\
	\textbf{Entry conditions}:
	\begin{enumerate}
		\item{User is logged in}
	\end{enumerate}
	\textbf{Event flow}:
	\begin{enumerate}
		\item{User enters the section "Explore data"}
    \item{The user chooses to get the map about streets with highest frequency of violations}
    \item{The system retrieves data from the database of violations, counting for each street the number of occurrencies}
    \item{The system sends to the external maps API the count of violation and the road name}
    \item{The app shows the map with an overlay which higlights the areas with a gradient color according to the number of vioations occurred}
	\end{enumerate}
	\textbf{Exit conditions}:
        \item{User wants to go back to "Explore data" area}
	\textbf{Exceptions}:
  \begin{enumerate}
    \item{If there are no records the app will report no data available message}
  \end{enumerate}
	\rule{\linewidth}{0.4pt}
  %%%%%%%%%%%%%%%%%%%%%%%%%%%%%%%%%%%%%%%%%%%%%%%%%%%%%%%%%%%%%%%%%%%%%%%
	\textbf{ID}: \ucas{5b} \\
	\textbf{Name}: Mine information by Authority - offenders \\
	\textbf{Actor}: AuthorityUser   \\
	\textbf{Entry conditions}:
	\begin{enumerate}
		\item{AuthorityUser is logged in}
	\end{enumerate}
	\textbf{Event flow}:
	\begin{enumerate}
		\item{AuthorityUser enters the section "Explore data"}
		\item{The system asks which kind of data the AuthorityUser wants to know}
    \item{The AuthorityUser chooses to get the data about veichles that committed the highest number of violations}
    \item{The system queries the table where for each licence plate is associated the count of violations }
    \item{The system will report in a tabular way the plate of the veichle and the count of violations committed}
    \item{If the AuthorityUser scrolls down, the system will offer the chance to load more rows}
	\end{enumerate}
	\textbf{Exit conditions}:
  \begin{enumerate}
    \item{AuthorityUser wants to go back to "Explore data" area}
  \end{enumerate}
	\textbf{Exceptions}:
	\begin{enumerate}
		\item{If there are no records the app will report no data available message}
	\end{enumerate}
	\rule{\linewidth}{0.4pt}
  %%%%%%%%%%%%%%%%%%%%%%%%%%%%%%%%%%%%%%%%%%%%%%%%%%%%%%%%%%%%%%%%%%%%%%%
  \textbf{ID}: \ucas{5c} \\
  \textbf{Name}: Mine information by EndUser - offenders \\
  \textbf{Actor}: EndUsers   \\
  \textbf{Entry conditions}:
  \begin{enumerate}
    \item{User is logged in}
  \end{enumerate}
  \textbf{Event flow}:
  \begin{enumerate}
    \item{User enters the section "Explore data"}
    \item{The system asks which kind of data the user wants to know}
    \item{The User chooses to get the data about veichles that committed the highest number of violations}
    \item{The system queries the table where for each licence plate is associated the count of violations }
    \item{the System associates an anonymized sequential identifier to each plate}
    \item{The system will report in a tabular way the identifier of the veichle and the count of violations}
    \item{If the User scrolls down the system will offer the chance to load more rows}
  \end{enumerate}
  \textbf{Exit conditions}:
  \begin{enumerate}
    \item{User wants to go back to "Explore data" area}
  \end{enumerate}
  \textbf{Exceptions}:
  \begin{enumerate}
    \item{If there are no records the app will report no data available message}
  \end{enumerate}
  \rule{\linewidth}{0.4pt}
  %%%%%%%%%%%%%%%%%%%%%%%%%%%%%%%%%%%%%%%%%%%%%%%%%%%%%%%%%%%%%%%%%%%%%%%
\item{Advanced function}\\
		\textbf{ID}: \ucas{6} \\
		\textbf{Name}: Ticket approval  \\
		\textbf{Actor}: AuthorityUser   \\
		\textbf{Entry conditions}:
		\begin{enumerate}
			\item{A new violation is inserted in database}
      \item{AuthorityUser logged in}
		\end{enumerate}
		\textbf{Event flow}:
		\begin{enumerate}
      \item{Every time a new violation is created by a EndUser the system will create automatically a ticket to be approved}
			\item{AuthorityUser enters the section "Tickets"}
			\item{AuthorityUser enters the section "Approve Tickets"}
      \item{The System will show the list of tickets available for approval }
      \item{AuthorityUser selects one tiket and system will show the related details}
      \item{System will ask the AuthorityUser if he wants to approve or not the ticket}
  		\end{enumerate}
		\textbf{Exit conditions}:
    \begin{enumerate}
      \item{User wants to go back to "Ticket" area}
      \item{AuthorityUser approves the ticket}
      \item{AuthorityUser doesn't approve the ticket}
    \end{enumerate}
		\textbf{Exceptions}:
		\begin{enumerate}
			\item{If there are no tickets pending, the app will report no data available message}
		\end{enumerate}
		\rule{\linewidth}{0.4pt}
    %%%%%%%%%%%%%%%%%%%%%%%%%%%%%%%%%%%%%%%%%%%%%%%%%%%%%%%%%%%%%%%%%%%%%%%
    \textbf{ID}: \ucas{5} \\
    \textbf{Name}: Ticket statistics selection \\
    \textbf{Actor}: AuthorityUser   \\
    \textbf{Entry conditions}:
    \begin{enumerate}
      \item{AuthorityUser logged in}
    \end{enumerate}
    \textbf{Event flow}:
    \begin{enumerate}
      \item {AuthorityUser enters the section "ticket statistics"}
      \item {The system will show the available ticket statisics options available to show}
    \end{enumerate}
    \textbf{Exit conditions}:
    \begin{enumerate}
      \item{the AuthorityUser chooses the kind of statistics he wants to see}
    \end{enumerate}
    \textbf{Exceptions}:
    \begin{enumerate}
    \end{enumerate}
    \rule{\linewidth}{0.4pt}
    %%%%%%%%%%%%%%%%%%%%%%%%%%%%%%%%%%%%%%%%%%%%%%%%%%%%%%%%%%%%%%%%%%%%%%%
    \textbf{ID}: \ucas{5} \\
    \textbf{Name}: Statistics - offenders \\
    \textbf{Actor}: AuthorityUser   \\
    \textbf{Entry conditions}:
    \begin{enumerate}
      \item{AuthorityUser logged in}
      \item{AuthorityUser has selected to see the statistics about offenders }
    \end{enumerate}
    \textbf{Event flow}:
    \begin{enumerate}
      \item{The system queries the table about all tickets, getting the count of tickets associated to every citizen present in the database of ticket created}
      \item{The system will report in a tabular way the name of the citizen and the count of approved tickets he has received}
      \item{If the AuthorityUser scrolls down, the system will offer the chance to load more rows}
    \end{enumerate}
    \textbf{Exit conditions}:
    \begin{enumerate}
      \item{the AuthorityUser wants to go back to other sections}
    \end{enumerate}
    \textbf{Exceptions}:
    \begin{enumerate}
    \end{enumerate}
    \rule{\linewidth}{0.4pt}

    %%%%%%%%%%%%%%%%%%%%%%%%%%%%%%%%%%%%%%%%%%%%%%%%%%%%%%%%%%%%%%%%%%%%%%%
    \textbf{ID}: \ucas{5} \\
    \textbf{Name}: Statistics - trends \\
    \textbf{Actor}: AuthorityUser   \\
    \textbf{Entry conditions}:
    \begin{enumerate}
      \item{AuthorityUser logged in}
      \item{AuthorityUser has chosen to see the Statistics - trend option}
    \end{enumerate}
    \textbf{Event flow}:
    \begin{enumerate}
      \item {
      \item
    \end{enumerate}
    \textbf{Exit conditions}:
    \begin{enumerate}

    \end{enumerate}
    \textbf{Exceptions}:
    \begin{enumerate}
    \end{enumerate}
    \rule{\linewidth}{0.4pt}
